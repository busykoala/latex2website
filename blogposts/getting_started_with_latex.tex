\documentclass{article}
  \usepackage{amsmath}
  \usepackage{listings}
  \usepackage[hidelinks]{hyperref}

  \begin{document}

  \section{Getting Started with LaTeX}

  LaTeX is great for many use cases. It makes it very handy to cite and add a
  bibliography, put references in a file or place elements without them jumping
  around randomly. Also it is pretty easy to work with formulas and equations.

  Other reasons to choose LaTeX are that it is platform independent and the
  changes can be tracked well in version control.

  Checkout \href{https://git.sr.ht/~busykoala/latex_article}{this repository}
  for the complete project.

  \subsection{How to structure my project}

  I usually create a similar structure as below for my documents. There is a
  directory to put the figures, another containing the literature file and the
  third for the glossary and acronym entries.
  The main.tex file contains the actual document. The Makefile is the
  configuration for building the document and the README.md documents how
  everything is done. For bigger projects I add a content folder with files for
  each section.

  \begin{lstlisting}[language=Bash]
  $ tree
    .
    ├─ figures
    │  └─ figure1.png
    ├─ bib
    │  └─ literature.bib
    ├─ glossary
    │  ├─ glossary.tex
    │  └─ acronym.tex
    ├─ main.tex
    ├─ Makefile
    └─ README.md
  \end{lstlisting}

  \subsection{Makefile - How to Build the Document}

  On Linux or MacOS we can define a makefile as below to compile the document
  on changing anything. It has a section to run all the compilation commands
  necessary to build the document, the bibliography and the glossary. Then
  there is a section to clean up and one to watch file changes and rerunning
  the latter.

  \begin{lstlisting}
    ALL=$(wildcard *.tex content/*.tex figures/*.{png,jpg})
    MAIN=main.tex
    LATEX=rubber --pdf
    SHELL=/bin/sh

    all:
      $(LATEX) $(MAIN)
      biber $(MAIN:.tex=)
      makeglossaries $(MAIN:.tex=)
      $(LATEX) $(MAIN)

    clean:
      rubber --clean $(MAIN)

    watch:
      @while [ 1 ]; do             \
        inotifywait $(ALL);        \
        sleep 0.01;                \
        make all;                  \
        make clean;                \
        echo "%%%%%%%%%%%%%%%%%%"; \
        echo "Finished Compiling"; \
        echo "%%%%%%%%%%%%%%%%%%"; \
        done
  \end{lstlisting}

  \subsection{Bibliography \& Glossary/Acronyms}

  The bibliography source and the definitions of glossary and acronym entries
  are located in separate files. The bibliography file can either be assembled
  by hand or exported from e.g. Zotero. \\
  Within the headers there is the hyperref package. It sets clickable
  references wherever there is a citation or reference to one of the
  glossaries.\\
  Then defining biblatex we choose between many citation styles (here IEEE).
  Since we use biber as the backend there are a whole lot of styles to choose
  from, probably including the one you have to use.

  \begin{lstlisting}
    \documentclass{article}
      \usepackage[hidelinks]{hyperref}
      \usepackage[style=ieee,backend=biber]{biblatex}
      \addbibresource{bib/literature.bib}
      \usepackage[acronym]{glossaries}
      \makeglossaries

      \begin{document}
      \section{First Section}
      Example Quotation \cite{aharonov2008quantum} \\
      Example Acronym \acrshort{ITSCM} \\
      Example Glossary \Gls{microservice} \\

      \printbibliography
      \input{glossary/glossary.tex}
      \printglossary
      \input{glossary/acronym.tex}
      \printglossary[type=\acronymtype]
      \end{document}
  \end{lstlisting}

  Glossary entries in glossary/glossary.tex start with the tag to be referenced
  with. Then they have a name and a description.

  \begin{lstlisting}
    \newglossaryentry{microservice}
    {
      name=Microservice,
      description={A microservice is a service to be used in a bigger context.}
    }
  \end{lstlisting}

  Acronym entries contain the tag to be referenced, the acronym and its
  meaning.

  \begin{lstlisting}
    \newacronym{ITSCM}{ITSCM}{IT Service Continuity Management}
  \end{lstlisting}

  Check out \href{https://www.overleaf.com/learn/latex/Glossaries}{Overleaf}
  to get more information about how to use glossaries.

  \subsection{Table of Content \& Appendix}

  Adding a table of content is a piece of cake. Also, adding the appendix is
  not rocket science either. The appendices are numbered A, B, C...
  automatically.

  \begin{lstlisting}
    \documentclass{article}
      \usepackage[toc,page]{appendix}

      \begin{document}
      \tableofcontents

      \section{First Section}
      Normal section

      \begin{appendices}
        \section{I'm Appendix A}
        Some text
        \section{I'm Appendix B}
        More text
      \end{appendices}
      \end{document}
  \end{lstlisting}

  \subsection{List of Figures \& Tables}

  Let's add some more Lists to the document. The included packages help to
  easily add figures and tables. Adding something in the figure or table
  section will automatically add it to it's corresponding list using the
  caption as it's title.

  \begin{lstlisting}
    \documentclass{article}
      \usepackage{graphicx}
      \usepackage{tabularx}
      \usepackage{float}

      \begin{document}

      \begin{figure}[H]
        \centering
        \includegraphics[scale=.5]{figures/example_figure.png} \\
        \caption{Example Figure}\label{fig:example_figure}
      \end{figure}

      \begin{table}[htbp]
        \centering
        \begin{tabularx}{\textwidth}{| p{1cm} | X | p{6cm} |}
          \hline
            1 & Lorem
              & Ipsum \\ \hline
            2 & Dolorum
              & Est \\ \hline
        \end{tabularx}
        \caption{Example Table}\label{tbl:example_table}
      \end{table}

      \listoffigures
      \listoftables
      \end{document}
  \end{lstlisting}

  The label lets us reference the table like:

  \begin{lstlisting}
    \autoref{fig:example_figure}
  \end{lstlisting}

  \subsection{Header \& Footer}

  A nice header and footer can be added using fancyhdr and resize the page a
  little using geometry. \\
  Also very handy is the automatically set date (in this setup on the footers
  left side).

  \begin{lstlisting}
    \documentclass{article}
      \usepackage[a4paper, total={6in, 9in}]{geometry}
      \usepackage{fancyhdr}
      \pagestyle{fancy}
      \usepackage{datetime}

      \newdateformat{monthdayyeardate}{
        \THEDAY. \monthname[\THEMONTH] \THEYEAR}

      \pagestyle{fancy}
      \fancyhf{}
      \rhead{Top right text}
      \lhead{Top left text}
      \rfoot{Page \thepage}
      \lfoot{\monthdayyeardate\today}

      \begin{document}
      \end{document}
  \end{lstlisting}

  \subsection{Language \& Encoding}

  The following packages can be used to set the encoding and a custom language.
  In this example we use the language ngerman. This will rename most of the
  headings to that language for example.

  \begin{lstlisting}
    \documentclass{article}
      \usepackage[utf8]{inputenc}
      \usepackage[ngerman]{babel}

      \begin{document}
      \end{document}
  \end{lstlisting}

  \subsection{Title Page \& Abstract}

  To add a title page add a new section within the document.

  \begin{lstlisting}
    \documentclass[titlepage]{article}

      \begin{document}
        \title{Example Document}
        \author{Matthias Osswald}
        \maketitle

        \begin{abstract}
          Lorem ipsum dolor sit amet, consetetur...
        \end{abstract}
      \end{document}
  \end{lstlisting}

  \subsection{Useful stuff}

  The emergency stretch option is very useful to avoid lots of box oversize
  errors since it allows a little over sizing. \\
  The showframe package shows the box frames on removing [noframe] which also
  is very handy for debugging.

  \begin{lstlisting}
    \documentclass{article}
      \emergencystretch=1em
      \usepackage[noframe]{showframe}

      \begin{document}
      \end{document}
  \end{lstlisting}

  To include external pdf pages into the document the package "pdfpages" can be
  used. For including some colors to tables "colortbl" is pretty nice.

  \end{document}
