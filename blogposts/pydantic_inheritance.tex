\documentclass{article}

\usepackage{amsmath}
\usepackage{listings}
\usepackage[hidelinks]{hyperref}

\lstset{numbers=left, numberstyle=sffamilytiny}

\begin{document}

  \section{Pydantic Models \& Inheritance}

  As a Python developer typing is not something which is commonly used. Still
  it is very nice to have when working with e.g. external interfaces. For basic
  type hinting there is the typing library in the standard lib. But it is very
  limited when it comes to validation and serialization/deserialization.

  Luckily there is this great library
  \href{https://pydantic-docs.helpmanual.io/}{Pydantic}.
  Below there is an example how to use it's most important features like
  serialization, deserialization, validation and inheritance.

  \subsection{}

  \begin{lstlisting}
from datetime import datetime
from typing import List

from pydantic import BaseModel
from pydantic import validator


class Wheel(BaseModel):
    brand: str
    date: datetime

    @validator('date')
    def date_must_be_after_2018(cls, date):
        due_date = datetime(2019, 1, 1, 0, 0)
        if date < due_date:
            raise ValueError('must be after 2018')
        return date


class Car(BaseModel):
    model: str
    wheels: List[Wheel]


if __name__ == '__main__':
    good_data = {
        'model': 'Ford Focus',
        'wheels': [
            {'brand': 'Michelin', 'date': '2020-12-22T22:22:00'},
            {'brand': 'Bridgestone', 'date': '2019-12-22T22:22:00'}
        ],
    }

    bad_data = {
        'model': 'Ford Focus',
        'wheels': [
            {'brand': 'Michelin', 'date': '2018-12-22T22:22:00'},
            {'brand': 'Bridgestone', 'date': '2019-12-22T22:22:00'}
        ],
    }

    my_fancy_two_wheeled_car = Car(**good_data)
    # this should return good_data with datetime values as wheel date
    my_fancy_two_wheeled_car.dict()

    # this should raise a pydantic.ValidationError because of one wheel date
    my_broken_two_wheeled_car = Car(**bad_data)

  \end{lstlisting}

\end{document}
