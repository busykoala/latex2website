\documentclass{article}

\usepackage{amsmath}
\usepackage{listings}
\usepackage[hidelinks]{hyperref}

\lstset{numbers=left, numberstyle=sffamilytiny}

\begin{document}

  \section{Teamwork using Overleaf and Zotero}

  If you need to work on a scientific paper as a team it might be hard to
  coordinate everything using LaTeX.
  A very cool solution for that is using Overleaf connected with Zotero.

  Overleaf lets your team work simultaneously on the LaTeX paper as you might
  know it from Google Docs.
  Another great advantage about it is, that you don't have to set up the complete
  LaTeX build environment locally.

  For your bibliography you can now create a shared group in Zotero.
  If you hit \textit{Groups > your-group > Settings} and set \textit{Library Reading}
  to \textit{Anyone on the internet} you will be able to use the Zotero api to
  add a synchronized bibtex file to Overleaf.

  For that go to your Overleaf project and select \textit{New File} and then
  \textit{From External URL}.
  Use something like the following pattern for the URL:

  \begin{lstlisting}
https://api.zotero.org/groups/<group-id>/items/top?format=bibtex&style=numeric&limit=100
  \end{lstlisting}

  You can retrieve the \textit{<group-id>} from the url when you visit it online on
  zotero.org.

\end{document}
